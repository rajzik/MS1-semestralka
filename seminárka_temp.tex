%----------------------------------------------------------------------------------------
%	Konfi begin  ( preabule dokumentu zač)
%----------------------------------------------------------------------------------------

\documentclass[paper=a4, fontsize=11pt]{scrartcl} 
\usepackage[T1]{fontenc} 
\usepackage{fourier}
\usepackage{url} 
\usepackage[utf8]{inputenc}
\usepackage[czech,english]{babel}
\usepackage{amsmath,amsfonts,amsthm} 
\usepackage{lipsum}
\usepackage{sectsty} 
\allsectionsfont{\centering \normalfont\scshape}
\usepackage{fancyhdr} 
\pagestyle{fancyplain} 
\fancyhead{} 
\fancyfoot[L]{}
\fancyfoot[C]{} 
\fancyfoot[R]{\thepage}
\renewcommand{\headrulewidth}{0pt}
\renewcommand{\footrulewidth}{0pt}
\setlength{\headheight}{13.6pt} 
\numberwithin{equation}{section}
\numberwithin{figure}{section} 
\numberwithin{table}{section}
\setlength\parindent{0pt} 
\usepackage[ddmmyyyy]{datetime}
\newdate{date}{12}{11}{2017}
\date{\displaydate{date}}
\newcommand{\lr}{LoRa™ }
\newcommand{\ara}{adruina }
\newcommand{\aru}{adruinu }
\newcommand{\Aro}{Adruino™ }
\newcommand{\aro}{adruino }
%----------------------------------------------------------------------------------------%	Konfi end  (preabule dokumentu kon)
%----------------------------------------------------------------------------------------

%----------------------------------------------------------------------------------------
%	Titulní strana
%----------------------------------------------------------------------------------------
\pagenumbering{gobble}
\newcommand{\horrule}[1]{\rule{\linewidth}{#1}} 
\title{	
	\normalfont \normalsize 
	\textsc{Mendelova univerzita v Brně\\Provozně ekonomická fakulta} \		\ [25pt] 
	\horrule{0.5pt} \\[0.4cm] 
	\huge Možnosti realizace technologie \lr \newline pomocí \Aro\\ 
	\horrule{2pt} \\[0.5cm] 
}
\author{Josef Havránek} % Vaše jméno
\date{\displaydate{date}} 
\begin{document}
\maketitle 
\newpage
%----------------------------------------------------------------------------------------
%	Úvod
%----------------------------------------------------------------------------------------
\section{Úvod}
	\pagenumbering{arabic}
	\setcounter{page}{1}
	V této práci se hodlám zabývat softwarovou a hardwarovou realizací
	připojení \ara do sítě \lr. Ač se to může zdát že jde odchýlení od
	tématu není tomu tak. Přímá  implementace na \aru pomocí SDR
	\footnote{Softwarově definované rádio} či bitbangovaní
	\footnote{bitbangování je přesně časované nastavování výstupů
	mikrokontroléru k softwarové emulaci  bez podpory hardware, je
	často psané v assembleru} vlastního hardwaru by byla obtížná. Ať už
	kvůli malému výkonu \ara vyšší spotřebě při přímém  softwarovém dekódování signálu či tomu že jde o patentovanou technologii.			\footnote{https://www.google.com/patents/US7791415} Budu se tedy
	zaměřovat na již vytvořené kompatibilní řešení v podobě prodávaných
	modemu/modulů. \\Dále chci podotknout že zapojení do sítě lze
	udělat dvěma způsoby lze si koupit službu od Českých
	radiokomunikací ( je ale nutno použít 868MHz)  která vás připojí po
	celé ČR nebo si postavit vlastní \lr bránu která funguje lokálně(lze použít jak 433MHz tak i 868MHZ v česku jsou to volná pásma.			\footnote{dle VO-R/10/11.2016-13}
%----------------------------------------------------------------------------------------
%	Hardware realizace
%----------------------------------------------------------------------------------------
\section{Možnosti hardware realizace}
	Základním předpokladem pro připojení jakéhokoliv modulu do \ara je kompatibilita, ať už jde o úroveň vysílaných signálů ve
	Voltech(pro \aro jak je to 3,3 a 5) či samotný sériový \uv{protokol} (\aro hardwarově podporuje SPI, $I^2C$ , sériovou linku a
	softwarově bylo bitbangováno mnoho dalších \uv{protokolů}). 
	\subsection{Shieldy}
		Shieldy pro \aro jsou desky tištěných spojů které se mají stejný profil a stejné uspořádání připojení jako \Aro. Jejich
		instalaceje naprosto jednoduchá, stačí je pouze zasunout v správné		orientaci \ara. Avšak nakonec jsou založeny na naprosto
		stejných modulech a proto jejich vlastnosti budu popisovat až v sekci 		moduly jelikož 	jsou na nich přímo závislé.\newline			\newline
		Výčet Shieldů pro \aro:
		\begin{itemize}
			\item Dragino \lr Shield (používá modul HopeRF RFM95W)
			\item \lr SHIELD FOR ARDUINO™ od firmy \uv{UDOIT s.r.l.}		(používá modul Aurel XTR 8LR100) 
		\end{itemize}
		\newpage
	\subsection{Moduly bez shieldu}
		Zde uvedu pouze seznam modulů s jejich důležitými parametry ( není 		li řečeno jinak pracovní napětí je vždy 3,3 Voltů.
		\begin{enumerate}
			\item Aurel XTR 8LR100\footnote{\url{http://www.aurelwireless.com/wp-content/uploads/user-manual/650201364G_um.pdf}}
				\begin{itemize}
					\item \textbf{rozhraní:} UART(9600;19200 a 115200 bps)
					\item  \textbf{frekvenční rozsah:} od 869,4 do 869,65 MHz 
					\item  \textbf{Software kompatibilita :} možná					kompatibilita u opensource knihovny od firmy UDUIT	
					\item \textbf{Cena:} 677,-Kč\footnote{dne 12/11/17				\url{https://www.tme.eu/cz/details/xtr-8lr100/komunikacni-moduly-rf-aurel/aurel/650201364g}} 
				\end{itemize}
			\item HopeRF RFM95W\footnote{\url{http://www.hoperf.com/upload/rf/RFM95_96_97_98W.pdf}}
				\begin{itemize}
					\item \textbf{rozhraní:} SPI
					\item  \textbf{frekvenční rozsah:} 868/915 MHz a 433MHz			pro verzi s označením RFM96W
					\item  \textbf{Software kompatibilita :} Arduino-LMIC
					\item \textbf{Cena:} 223,-Kč\footnote{dne 12/11/17				\url{https://www.tme.eu/cz/details/rfm95w-868s2/komunikacni-moduly-rf/hope-microelectronics}} 
				\end{itemize}
			\item NiceRF LoRa1276\footnote{\url{http://bit.ly/2yrppJ5}}
				\begin{itemize}
					\item \textbf{Cena:} 523,-Kč\footnote{dne 12/11/17				\url{https://arduino-shop.cz/arduino/1486-iot-868mhz-lora-lpwan-modul-1276-cra-kompatibilni-1476993740.html}}
				\end{itemize}
				jinak stejné parametry Jako modul od HopeRF
			\item Modtronix inAir9/9B\footnote{\url{http://modtronix.com/inair9.html}}
				\begin{itemize}
					\item \textbf{Cena:} 307,-Kč\footnote{dne 12/11/17		 \url{http://modtronix.com/inair9.html?sef_rewrite=1&currency=EUR}}
				\end{itemize}
				jinak stejné parametry Jako modul od HopeRF
		\end{enumerate}
\newpage
%----------------------------------------------------------------------------------------
%	Sofware Realizace
%----------------------------------------------------------------------------------------
\section{Možnosti software realizace}
	Věcí číslo jedna při vlastní softwarové implementaci je trpělivost a čtení dokumentace. Chci také ještě dodat že používání
	knihovny není nezbytnost, ulehčuje to však získávaní dat z modulů a většinou umožňuje i jejich ovládání a řízení 
		\subsection{Knihovna Arduino-LMIC}
		Tato knihovna je jedná veřejně dostupná (na githubu) knihovna která je 	pravidelně udržována ( poslední update 30/8/17). Tato
		knihovna však obsahuje několik neotestovaných, věcí avšak většinu důležitých věcí, 		jako šifrování, příjímání paketů v okně
		RX2, bezdrátová aktivace či	sestavení a rozbalení \lr WAN paketů, by měla umět. Tato knihovna spolupracuje s jakýmkoliv modulem
		který s většinou modulů komunikující přes rozhraní SPI.\footnote{https://github.com/matthijskooijman/arduino-lmic}
%----------------------------------------------------------------------------------------
%	Závěr
%--------------------------------------------------------------------------------------
\section{Závěr}
	Připojení \ara do sítě \lr není až tak složité, pomineme li tedy připojení do sítě Českých radiokomunikací to jsem nezkoumal.
	Navíc \aro muže být docela flexibilní, muže fungovat jako vzdálený odesílač z hromady senzorů ke kterým může být připojen či muže
	fungovat i jako mezi most mezi LPWAN\footnote{low power WAN do které se \lr řadí} a klasickou internetovou sítí, AP
	\footnote{aplikační programové rozhraní} je na to dost a pokud by to nestačilo lze si udělat vlastní na bázi Dropboxu, Google Drive
	či něčeho úplně jiného. Pokud bych měl vzít v potaz reálnou implementaci tak nejtěžší na zapojení většího počtu \Aro desek bude
	hromadná agregace dat a hromadné vyhodnocování včetně detekce chyb senzorů.

\end{document}
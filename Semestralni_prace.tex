%----------------------------------------------------------------------------------------
%	Konfigurace dokumentu
%----------------------------------------------------------------------------------------

\documentclass[paper=a4, fontsize=11pt]{scrartcl} 
\usepackage[T1]{fontenc} 
\usepackage{fourier} 
\usepackage[utf8]{inputenc}
\usepackage[czech, english]{babel}
\usepackage{amsmath,amsfonts,amsthm} 
\usepackage{lipsum}
\usepackage{sectsty} 
\allsectionsfont{\centering \normalfont\scshape}
\usepackage{fancyhdr} 
\pagestyle{fancyplain} 
\fancyhead{} 
\fancyfoot[L]{}
\fancyfoot[C]{} 
\fancyfoot[R]{\thepage}
\renewcommand{\headrulewidth}{0pt}
\renewcommand{\footrulewidth}{0pt}
\setlength{\headheight}{13.6pt} 
\numberwithin{equation}{section}
\numberwithin{figure}{section} 
\numberwithin{table}{section}
\setlength\parindent{0pt} 
\usepackage[ddmmyyyy]{datetime}
\newdate{date}{26}{09}{2017}
\date{\displaydate{date}}

\addto\captionsenglish{% Replace "english" with the language you use
\renewcommand{\contentsname}%
  {Obsah}%
}

%----------------------------------------------------------------------------------------
%	Titulek
%----------------------------------------------------------------------------------------

\newcommand{\horrule}[1]{\rule{\linewidth}{#1}} 
\title{	
\normalfont \normalsize 
\textsc{Mendelova univerzita v Brně\\Provozně ekonomická fakulta} \\ [25pt] 
\horrule{0.5pt} \\[0.4cm] 
\huge Technologie Wi-Fi HaLow (založena na standardu IEEE 802.11ah) \\ % Název semestrální práce
\horrule{2pt} \\[0.5cm] 
}

\author{Jan Šilhan} % Vaše jméno

\date{\displaydate{date}} 

\begin{document}

\maketitle 

\newpage

\tableofcontents

\newpage

%----------------------------------------------------------------------------------------
%	Úvod
%----------------------------------------------------------------------------------------

\section{Úvod}

V této práci se budu zabývat Wi-Fi a přenosem dat pro IoT \footnote{Internet of Things}, 
konkrétně Wi-Fi IEEE 802.11ah (HaLow).
Přenos dat v IoT je nedílnou součástí, ale otázka energetické náročnosti přenosu dat s ní.
Dále také přenos na větší vzdálenosti. 
Doposud zde je mainstreamovým medii pro přenos bluetooth a Wi-Fi. 
Bluetooth je energeticky nenáročný ovšem dosah není tak uspokojivý, 
Wi-Fi na druhé straně muže přenášet velké množství dat na relativně velké vzdálenosti, 
ale s tím spojená cena energetické náročnosti. 
Dnes je však tento problém z velké části vyřešen, s příchodem IEEE 802.11ah. 

\section{Historie}
Wi-Fi a standard IEEE 802.11 je poprvé navrhnut a publikován v roce 1997. 
Rychlost tohoto standartu mohla dosahovat až 2 Mb/s na vzdálenost circa 20 metrů uvnitř budovy, 
venku až 100 metrů.
V tomto standartu byla také nadefinovaná přenosová frekvence 2.4 GHz která se používá do dnes.
	\subsection{IEEE 802.11a}
	V roce 1999 byli představeny dvě revize standartu. Jedna z nich je 
	IEEE 802.11a který začal používat novou přenosovou frekvenci 5 GHz a 3.7 GHz. 
	Zvýšení frekvence přineslo větší datovou prostupnost, 
	ale v budovách přichází problém s prostupností zdí. 

	\subsection{IEEE 802.11b/g/n}
	Dodnes používané protokoly b/g/n, 
	jsou nejčastěji viděny v domácnostech, 
	některé starší zařízení do dnes využívá protokol b z roku 1999. 
	Avšak většina vysílačů dovoluje použít všechny výše zmíněné protokoly. 
	Při ideálních podmínkách je přenosová rychlost až 600 Mb/s.

	\subsection{IEEE 802.11ac}
	V roce 2013 byl představen nový standart IEEE 802.11ac
	 který je teoreticky schopný přenosové rychlosti až 3466.8 Mb/s. 
	 Dnes je hojně využívaný pro bezdrátový internet, to ovšem způsobuje rušení a degradaci signálu,
	 Brno je toho velkým příkladem.

\newpage
	 
\section{IEEE 802.11ah}
Na přelomu roku 2017 je představen nový standart IEEE 802.11ah, 
který je průlomový svou frekvencí, 
která na rozdíl od předchozích verzí je nižší konkrétně 900 MHz. 
Výhody nižší frekvence jsou zjevné, 
hmotné objekty jsou daleko propustnější, 
dále je možné vysílat na delší vzdálenosti bez ztráty kvality. 
Naopak tento standart není plánován pro velké objemy dat, 
i když datový tok není problém, 
životnost baterie ano. 
Tento standart je navržen pro malé objemy dat, 
který nebude tak častý jako internetový provoz. \\
Nový standart slibuje menší spotřebu, 
tím se dostáváme k tématu IoT 
které bude nyní možné spojovat skrze Wi-Fi.
	\subsection{Wi-Fi v IoT}
	Navržený pro chytré domácnosti a je odpovědí na Bluetooth, 
	který převládal v IoT dlouhou dobu, 
	ale už dnes je Wi-Fi v některých zařízeních jako například v kamerách 
	a jednoúčelových zařízeních. 
	Možné využití je ve všech IoT odvětvích. 
	\subsection{Dostupnost}
	Dnes je Wi-Fi HaLow dostupná velmi ojedinělá. Firma která využívá této technologie je LoRa Alliance\footnote{https://www.lora-alliance.org/}.
	Velká expanze je očekávána v roce 2018. 

\section{Závěr}
Nízká energetická náročnost je obecně trend a budoucnost technologie. 
Jakmile se IEEE 802.11ah stane mainstreamovou platformou nebude cesty zpět. 
I když toto řešení není konečné ani dokonalé, tak je to veliký krok ku předu. 
Wi-Fi HaLow má několik nedokonalostí, 
ale jediný konkurent je Bluetooth 5 a jen čas ukáže která platforma bude lepší.


\end{document}